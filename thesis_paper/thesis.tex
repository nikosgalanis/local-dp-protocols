% demo.tex
%
% Enjoy, evolve, and share!
%
% Compile it as follows:
%   latexmk
%
% Check file `dithesis.cls' for other configuration options.
%
\documentclass[inscr,ack,preface]{dithesis}

%\usepackage{graphicx}

%%%%%%%%%%%%%%%%%%%%%%%%%%%%%%%%%%%%%%%%%%%%%%%%%%%%%%%%%%%%%%%%%%%%%%%%%%%%%%%
%%%%%%%%%%%%%%%%%%%% User-specific package inclusions %%%%%%%%%%%%%%%%%%%%%%%%%
%%%%%%%%%%%%%%%%%%%%%%%%%%%%%%%%%%%%%%%%%%%%%%%%%%%%%%%%%%%%%%%%%%%%%%%%%%%%%%%
\usepackage{booktabs}
\usepackage{hyperref}
\usepackage{lipsum}
\usepackage{enumerate}
\usepackage{amsmath}
\usepackage{amssymb}
\usepackage{listings}

\hypersetup{
    unicode=true,                     % non-Latin characters in bookmarks
    pdffitwindow=true,                % page fit to window when opened
    pdfnewwindow=true,                % links in new window
    pdfkeywords={},                   % list of keywords
    colorlinks=true,                  % false: boxed links; true: colored links
    linkcolor=black,                  % color of internal links
    citecolor=black,                  % color of links to bibliography
    filecolor=black,                  % color of file links
    urlcolor=black,                   % color of external links
    pdftitle={},                      % title
    pdfauthor={},                     % author
    pdfsubject={}                     % subject of the document
}
%%%%%%%%%%%%%%%%%%%%%%%%%%%%%%%%%%%%%%%%%%%%%%%%%%%%%%%%%%%%%%%%%%%%%%%%%%%%%%%
%%%%%%%%%%%%%%%%%%%% User-specific package inclusions %%%%%%%%%%%%%%%%%%%%%%%%%
%%%%%%%%%%%%%%%%%%%%%%%%%%%%%%%%%%%%%%%%%%%%%%%%%%%%%%%%%%%%%%%%%%%%%%%%%%%%%%%


%%%%%%%%%%%%%%%%%%%%%%%%%%%%%%%%%%%%%%%%%%%%%%%%%%%%%%%%%%%%%%%%%%%%%%%%%%%%%%%
%%%%%%%%%%%%%%%%%%%%%% User-specific configuration %%%%%%%%%%%%%%%%%%%%%%%%%%%%
%%%%%%%%%%%%%%%%%%%%%%%%%%%%%%%%%%%%%%%%%%%%%%%%%%%%%%%%%%%%%%%%%%%%%%%%%%%%%%%
%%%%%%%%%%%%%%%%%%%%%%%%%%%%%%%%%%%%%%%%%%%%%%%%%%%%%%%%%%%%%%%%%%%%%%%%%%%%%%%
%%%%%%%%%%%%%%%%%%%%%% User-specific configuration %%%%%%%%%%%%%%%%%%%%%%%%%%%%
%%%%%%%%%%%%%%%%%%%%%%%%%%%%%%%%%%%%%%%%%%%%%%%%%%%%%%%%%%%%%%%%%%%%%%%%%%%%%%%


%%%%%%%%%%%%%%%%%%%%%%%%%%%%%%%%%%%%%%%%%%%%%%%%%%%%%%%%%%%%%%%%%%%%%%%%%%%%%%%
%%%%%%%%%%%%%%%%%%%%%%%%%%% Required Metadata %%%%%%%%%%%%%%%%%%%%%%%%%%%%%%%%%
%%%%%%%%%%%%%%%%%%%%%%%%%%%%%%%%%%%%%%%%%%%%%%%%%%%%%%%%%%%%%%%%%%%%%%%%%%%%%%%
%
% First name, last name
%
\authorFirstGr{Νικόλαος}
\authorFirstAbrGr{Ν.} % abbreviation of first name
\authorMiddleGr{Γ.}   % abbreviation of father's first name
\authorLastGr{Γαλάνης}
\authorFirstEn{Nikolaos}
\authorFirstAbrEn{N.}
\authorMiddleEn{G.}
\authorLastEn{Galanis}
\authorSn{1115201700019}

%
% The title of the thesis
%
\titleEn{Applications of Global and Local Differential Privacy}
\titleGr{Εφαρμογές της Καθολικής και Τοπικής Διαφορικής Ιδιωτικότητας}

%
% Month followed by Year
%
\dateGr{ΙΟΥΝΙΟΣ 2021}
\dateEn{JUNE 2021}

%
% Supervisor(s) info
%
\supervisorGr{Κωνσταντίνος Χατζηκοκολάκης}{Αναπληρωτής Καθηγητής}
\supervisorEn{Konstantinos Chatzikokolakis}{Associate Professor}

%
% Abstract, synopsis, inscription, ack, and preface pages.
%
% \setlength\parindent{24pt}

\abstractEn{
\par The problem of preserving privacy while extracting information during data analysis, has been an everlasting one. Specifically, during the big-data era, user details can be easily compromised by a malicious handler, something considered both as a security, and as a privacy issue.
\par With that being the case, there is a simple solution of denying the access to user data, thus making the mining of useful information about a plethora of subjects impossible. On the other hand, a successful mechanism would be for the data to be flowing without control, something that would be beneficiary for the advance of sciences (because of the huge amount of information that would be available), but a significant compromisation for the individuals' privacy. \par
However, none of these two solutions are applicable and helpful for solving our problem. The answer is finding a balance, that would benefit both parties: the users and their privacy, as well as the researchers. The optimal fix to the subject, is Differential Privacy, which is actually a promise, made by the data handler to the user, that they will not be affected, by allowing their data to be used in any analysis, no matter what other studies/databases/info resources are available. Meanwhile, the output data statistics should be accurate enough for any researcher to extract useful information from them.\par
This is a promise that in the first sight, seems rather hard to be achieved. Despite that, during this thesis, we will look closely into the theory which makes this form of privacy possible, by the addition of random noise to the user data. Differential Privacy is based on probabilistic theories, well known from the $20^{th}$ century, however, it is a rather new technique, which has yet to be fully implemented in a handy way for all data-miners to use.
\par The goal of this thesis, is to examine and compare previously created mechanisms for D.P., while also creating our own mechanism, that serves to the purpose of achieving Local D.P., a form of Differential Privacy that is nowadays widely used in machine learning algorithms, aiming to protect the individuals that send their personal data for analysis. We will do so, by creating a library that is easy to use, and applies to all the rules of data privacy, and then extract conclusions from its use.
}
\abstractGr{
Το πρόβλημα της διατήρησης της ιδιωτικότητας κατά την ανάλυση δεδομένων, υφίσταται για πολύ καιρό. Συγκεκριμένα, στην εποχή των big-data, λεπτομέρειες των χρηστών μπορούν εύκολα να παραβιαστούν από κακόβουλους χειριστές των δεδομένων, γεγονός που θεωρείται ζήτημα τόσο όσον αφορά την ασφάλεια, όσο και την προστασία της ιδιωτικότητας του ατόμου.\par
Mε την υπάρχουσα κατάσταση, υπάρχει η απλή λύση της άρνησης της πρόσβασης σε δεδομένα χρηστών, στον βωμό της προστασίας τους, κάτι που καθιστά την εξαγωγή συμπερασμάτων για ποικίλα θέματα αδύνατη. Από την άλλη, ένας επιτυχημένος μηχανισμός θα ήταν η ελεύθερη διακίνηση των δεδομένων, χωρίς φιλτράρισμά τους, γεγονός που θα ήταν ωφέλιμο για την πρόοδο των επιστημών (λόγω του μεγάλου όγκου δεδομένων που θα ήταν διαθέσιμος), αλλά μία μεγάλη παραβίαση της ιδιωτικότητας των ατόμων. 
\par 
Ωστόσο, καμία από τις δύο αυτές λύσεις δεν μπορεί να εφαρμοστεί και να μας βοηθήσει στην επίλυση τους προβλήματός μας. Η απάντηση είναι η εύρεση μίας ισορροπίας, η οποία ευνοεί και τα δύο μέρη: τους χρήστες και την ιδιωτικότητά τους, όπως και τους ερευνητές. Η βέλτιστη επίλυση του θέματος, είναι η Διαφορική Ιδιωτικότητα, που στην πραγματικότητα πρόκειται για μία υπόσχεση από τον χειριστή των δεδομένων προς τον χρήστη, πως ο χρήστης δεν θα επηρεαστεί αν επιτρέψει τη χρήση των δεδομένων του σε κάποια ανάλυση, χωρίς περιορισμούς όπως η παράλληλη ύπαρξη άλλων μελετών/βάσεων δεδομένων/πληροφοριών που υπάρχουν για αυτόν. Παράλληλα, τα στατιστικά του αποτελέσματος της ανάλυσης, πρέπει να είναι αρκετά ακριβή, ώστε ο ερευνητής να μπορεί να εξάγει χρήσιμη πληροφορία από αυτά. \par 
Η υπόσχεση αυτή, δείχνει δύσκολα υλοποιήσιμη με την πρώτη ματιά. Παρόλα αυτά, σε αυτήν την πτυχιακή εργασία, θα ερευνήσουμε με λεπτομέρεια τη θεωρία που καθιστά εφικτή αυτή τη μορφή ιδιωτικότητας, με την προσθήκη τυχαίου θορύβου στα δεδομένα. Η Διαφορική Ιδιωτικότητα βασίζεται σε πιθανοτικές κατανομές, γνωστές ήδη από τον $20^o$ αιώνα, όμως παραμένει μία νέα τεχνική, η οποία δεν έχει πλήρως υλοποιηθεί με τρόπο τέτοιον ώστε να μπορεί να χρησιμοποιηθεί από πολλούς ανθρώπους που είναι υπεύθυνοι για την εξαγωγή δεδομένων.
\par Σκοπός αυτής της πτυχιακής εργασίας, είναι να μελετήσουμε και να συγκρίνουμε ήδη υλοποιημένους μηχανισμούς πανω στην Δ.Ι., ενώ παράλληλα θα δημιουργήσουμε τον δικό μας μηχανισμό, ο οποίος χρησιμοποιείται για τους σκοπούς της Τοπικής Διαφορικής Ιδιωτικότητας που χρησιμοποιείται την σήμερον ημέραν σε αλγορίθμους μηχανικής μάθησης, με στόχο να προστατέψει τα δεδομένα που αποστέλλουν για εκμάθηση οι χρήστες. Θα το κατορθώσουμε αυτό δημιουργώντας μία προγραμματιστική βιβλιοθήκη η οποία είναι εύκολη στη χρήση, ικανοποιώνατας παράλληλα τους κανόνες της προστασίας δεδομένων, και τέλος θα εξάγουμε συμπεράσματα από τη χρήση της βιβλιοθήκης αυτής.

}
\acksEn{
\begin{greek}
Στη σελίδα αυτή αναφέρονται οι ευχαριστίες. Η σελίδα αυτή είναι προαιρετική. Παρατίθεται παράδειγμα ευχαριστιών.

Για τη διεκπεραίωση της παρούσας Πτυχιακής Εργασίας, θα θέλαμε να ευχαριστήσουμε τους επιβλέποντες, αν. καθ .Ευστράτιο Γεωργιάδη, Γρηγόριο Παπάμαλο, Αναστασία Γούσιου, Ξενοφών Παπαδόπουλο, για τη συνεργασία και την πολύτιμη συμβολή του στην ολοκλήρωση της.
\end{greek}
}
\prefaceEn{
\begin{greek}
Στον πρόλογο αναφέρονται θέματα που δεν είναι επιστημονικά ή τεχνικά, όπως το πλαίσιο που διενεργήθηκε η εργασία, ευχαριστίες, ο τόπος διεξαγωγής κλπ.
\end{greek}
}

\inscriptionEn{\emph{Στη σελίδα αυτή αναφέρονται οι αφιερώσεις. Η σελίδα αυτή είναι προαιρετική.}}

%
% Subject area and keywords
%
\subjectAreaGr{Προστασία και Ιδιωτικότητα Δεδομένων}
\subjectAreaEn{Data Privacy}
\keywordsGr{Διαφορική Ιδιωτικότητα, Ασφάλεια, Δεδομένα Χρηστών, Προστασία Δεδομένων, Θόρυβος σε Δεδομένα, Συλλογή Δεδομένων}
\keywordsEn{Differential Privacy, Security, User data, Data Privacy, Noisy Data, Aggregation of Data}

%
% Set the .bib file containing your paper publications (leave the extension out)
%
% This is optional, but it should be specified when option 'lop' is passed to
% the document class.
%
% Then, inside the document environment, you may use the command '\nocitelop' to
% site your papers, as you would traditionally do with the commands '\cite' or
% '\nocite'.
%
% The papers are printed in reverse chronological order.
%
%\lopfile{mypapers/pubs}
%%%%%%%%%%%%%%%%%%%%%%%%%%%%%%%%%%%%%%%%%%%%%%%%%%%%%%%%%%%%%%%%%%%%%%%%%%%%%%%
%%%%%%%%%%%%%%%%%%%%%%%%%%% Required Metadata %%%%%%%%%%%%%%%%%%%%%%%%%%%%%%%%%
%%%%%%%%%%%%%%%%%%%%%%%%%%%%%%%%%%%%%%%%%%%%%%%%%%%%%%%%%%%%%%%%%%%%%%%%%%%%%%%

\begin{document}

\frontmatter

\mainmatter

\chapter{INTRODUCTION}

\section{Need for Privacy}


\par In our days, data is everywhere, including our smartphones, our computers our TVs, even our watches. Every device and nearly every website track down data, in order to provide more personalized services. This, of course, is desired by the users, as they are more likely to see relevant advertisements, and in general, have a more unique experience while they are using their devices.

\par At the same time, the services that track down the data are also benefited, because of the way that science function: Experiments need to be made, thus the more available data in order to conduct them, the better. As an example, we might think the medical community: when someone logs-in to the hospital, it is beneficiary for the doctors to gather his data, in order to study his decease, and his potential recovery, not only for the shake of the patient, but also for the further study of his decease. 

\par While providing data may seem inevitable and yet beneficiary for all parties, there is always a risk that this data will be used in order to compromise the user's privacy. When the information lands in the wrong hands, it can expose some characteristics of the user that he does not want to be shared. In our medical example, let's now consider a patient with a rare decease, who logs-in to a local hospital. He might consent to share his personal data (name, age etc.), but only for the doctors to use it. What will happen when the doctors give the data of the whole hospital for analysis? This patient, considering he is one of the few that has this illness, may be stigmatized,  when the data analysts find out his condition. Wouldn't it be better for him, if, let's say, his name was not exposed? We will see later on, why this approach, is found to be successful, but not enough, for extreme cases.

\section{Definition of the problem of privacy}
\par In general, when we consider the \textbf{problem of privacy}, we refer to the protection of the disclosure of sensitive information of individuals, when a collection of data about these individuals (dataset) is made publicly available.

\subsection{Achieving Privacy via Anonymization}

\par One of the first, and rather successful attempts for preserving privacy, was anonymization, meaning removing all personal identifiers from the dataset. This technique is further developed, using famous algorithms like k-anonymity, l-diversity etc. However, there are several problems with this approach. Firstly, there are very computational heavy, as their complexity rises up to an exponential one, making the anonymization of a large dataset very slow. Also, the anonymization does not guarantee that the user will remain private, if other datasets are not anonymized. Let's once again consider our example. Suppose our patient goes to two separate hospitals for his treatment, and one of them uses the best anonymization techniques, while the other one provides the data without any form of privacy. Our patient is on both of the datasets, thus the techniques used by the first hospital are now useless. This expands to the real world, because, no matter how careful you (and the services that you use) are, a single data breach is enough for you to be compromised. 

\par So, right now, things seem a bit pessimistic, supposing that anonymization, no matter how well performed, can not fix our problem. An other successful technique, that is used on many other fields, is the addition of noise. Later on, we are going yo examine in which ways it can benefit us while trying to solve our problem.

\subsection{Achieving Privacy via Randomization}

Randomization can be applied to the data of the users in 2 different forms:
\begin{itemize}
    \item Apply random noise \textbf{directly to the data}. This will result to altered data, which will then be processed, so that the adversary will not be able to individualize the entries in the dataset.
    \item Apply random noise to \textbf{queries asked to the dataset}. In that case, the dataset is not directly available to the analysts. Instead, they are allowed to ask questions to the dataset, and the answers are then being randomized, and returned.
\end{itemize}

Both of the above approaches are utilized, but the second one is widely preferred. During our explorations of data privacy, we are going to dive in both of those techniques, as well as the libraries that they are used in.

\par As we can see, the randomization method looks good in theory, but we must answer to several questions before implementing it, such as: How can we define privacy for noisy queries? What type of noise do we need? We are going to answer those questions later on, during our next chapters. 

\section{Goal of this thesis}
It is made clear from our introduction, that the most effective up-to date method for applying privacy into a dataset, is via randomization. The method used, is called \textbf{Differential Privacy}, and is based on injecting noise into the users' data. The theory behind this method includes many mathematical theorems, thus, it can by easily explained. We will proceed by taking a look on those principles, and analysing the theory behind this form on data privacy. Then, we will proceed by examining some existing applications of D.P., especially some libraries that help us to apply this technique in a dataset. Finally, we are going to create our own library in order to apply Local D.P., a form of privacy that we will discuss in the next chapter. 


\chapter{PRINCIPLES OF DIFFERENTIAL PRIVACY}

During this chapter we are going to introduce the term of D.P., and its definition, alongside with the principles that need to be followed while applying it.

\section{Promise of Differential Privacy}

\par Differential Privacy is actually a promise made by the data handlers, to the participants of a study: "You will not be affected, adversely or otherwise, by allowing your data to be used in any study or analysis, no matter what other studies/ datasets/ info resources are available". 
\par The goal is to make the data widely available for analysis, while protecting the users. However, is it possible to learn nothing about an individual, while gathering useful information about a population? This is actually what D.P. is trying to achieve.


\section{Definition of Differential Privacy}
Before defining D.P., we must analyze some of the basic components of its definition.

\subsection{Randomized Response}
Randomized response is one of the earliest privacy mechanisms, that is used to conduct surveys where taboo behaviour is studied. The participants in those surveys are asked to answer truthfully, while they do not want to be stigmatized. There is a micro-world of what we are trying to achieve, thus we are going to give the algorithm of the randomized response in order to answer a binary (yes/no) question.

\begin{itemize}
    \item Flip a coin.
    \item If it lands on heads, answer truthfully
    \item If it lands on tails, flip another one
    \item If it lands on heads, answer no, else, answer yes
\end{itemize}

We are going to analyze this algorithm and its success in later chapters, but for now, it is enough to know that there exists a simple mechanism that adds noise, and is rather accurate for large samples.

Before giving the definition of D.P., we must define its components. 

\begin{itemize}
    \item \textbf{Probability Simplex}, given a discrete set $B$, is denoted as $\Delta(B)$ and is defined to be:
    \begin{align*}
        \Delta(B) = \{ x\in R^{|B|}: x_i \geq 0 \text { } \forall i \text{ and } \sum_{i=1}^{|B|} x_i = 1\}
    \end{align*}
    \item A \textbf {Randomized algorithm} $M$ with domain $A$ and discrete range of results $B$, is associated with the mapping $M: A\rightarrow\Delta(B)$. On
    \item \textbf{Distance between Databases:} The $l_1$ norm of a database x is denoted $||x||_1$ and it is defined to be: $||x||_1 = \sum_{i = 1}^{|x|} |x_i|$. Thus, the $l_1$ distance between 2 databases $x$ and $y$, is $||x-y||_1$, and the size of a database $x$ os $||x||_1$.
    
\end{itemize}
There are plenty definitions for D.P., but throughout this thesis we are going to use the one bellow.

\subsection{Definition}
Differential Privacy is defined as following:
\\
\\
A randomized algorithm $M$ with domain $N^{|x|}$ is (ε,δ)-differentially private, if for all $S \in Range(n)$ and for all $x,y \in N^{|x|}$ s.t. $||x - y||_1 \leq 1$
$$ Pr[M(x) \in S] \leq e^\epsilon Pr[M(y) \in S] + \delta$$

where the probability space is over the coin flips of the mechanisms $M$. If $\delta = 0$, we say that $M$ is ε-differentially private.

\\
It should be noted that D.P. is rather a definition than a strict algorithm. While relying on the definition of D.P., we can create different algorithms, which will all ensure that the result will be deferentially private. This allows us to create different forms of D.P., that will be analyzed later on this thesis.

The whole point of Differential Privacy, is that the output of a D.P. mechanism, should by \textbf{independent} of whether or not an individual is present in the domain $N$. The "ability" of the adversary to recognise the existence of a column in the dataset, is regularized by epsilon.

\section{The meaning of epsilon}
It is made clear from the above definition, that if we have a computational task, we might find different algorithms for applying D.P., but the result will always be of the same form: each user of the dataset, will get ε-D.P.. But what does the epsilon parameter actually mean?

By reading the mathematical equation, we observe that the higher the value of epsilon, the bigger the difference between the two probabilities (minimum and maximum). Thus, we extract the following statement about the value of epsilon during the application of Differential Privacy:

\begin{itemize}
    \item The \textbf{lower the epsilon} value, the \textbf{higher the privacy} guarantees for the users of the dataset.
    \item The \textbf{higher the epsilon} value, the \textbf{more accurate the results} produced.
\end{itemize}

In practice, epsilon values vary in the range $(0,5]$, as lower values are prohibited, and higher values are considered extreme cases. However, as mentioned in (TODO: Insert reference for Cynthia),  when epsilon is small, failing to be (ε,0)-differentially private is not necessarily alarming, if our algorithm is linearly increasing with ε (ex (2ε,0)-D.P). This happens because of the nature of the epsilon parameter, which guarantees very strict boundaries between databases. However, when ε increases by a lot, users' privacy suffers. 

In \textbf{Figure 2.1}, we can see in general terms, the function between the epsilon and the accuracy error, as well as the protection guaranteed. We will discuss in later sections the details on how these graphs are created, but now is a good time to get an overall picture of the accuracy error produced when applying D.P.
\bigskip
\bigskip\bigskip

\begin{figure}[!htb]\centering
      \includegraphics[width=0.6\textwidth]{images/epsilon_intro_graph.png}
  \caption{Accuracy Error as a function of epsilon}
\end{figure}

\section{Different forms of Differential Privacy}

As mentioned during the definition, due to the room that is left for its interpretation, there can be many forms of Differential Privacy. There are two major fields recognized, the \textbf{Global D.P.} and the \textbf{Local D.P.}.

Their major difference is the curator of the data. In the Global model, the curator must be trusted, as he collects the non-private data and has to pass them through a D.P. algorithm.

On the other hand, in the Local model, the curator may as well be untrusted, since the users perturb their data on their own, using a specific protocol. The key differences of the two forms are shown in the \textbf{Figure 2.2} below.

An other difference between the two models, is the amount of noise added. With the absence of a trusted curator, the users themselves must add a significant amount of noise into their data, in order to preserve their privacy. This of course results into a need of many users (several thousands), in order for the L.D.P. protocols to function correctly and accurately.


\begin{figure}[!htb]\centering
      \includegraphics[width=0.3\textwidth]{images/local_vs_global.png}
  \caption{Differences between LDP and GDP}
\end{figure}

In this thesis, we are going to examine both models, by quoting their definitions, observing already-existing algorithms, and creating our own L.D.P. protocol.

\section{Existing Problems of D.P.}

As every new step in Computer Science, Differential Privacy has some issues that are yet to be solved, and some others not covered by its definition. 

One major problem is the behaviour of the protocols \textbf{when the number of users is limited}. The definition of D.P. is based on the alteration of the data in order not to reveal sensitive information. Thus, if a small amount of users are involved in those protocols, the accuracy of the results might be way off the standards that we set, in order to satisfy the epsilon requirements of the user.

Another (unsolvable) issue, that mainly lies on the basis of surveys, is  \textbf{the possibility that conclusions drawn from a survey may reflect statistical information about an individual}.

For example, if a survey about the correlation of smoking and dental problems is conducted, someone that has specific dental problems might be deemed as a smoker, despite keeping his privacy about the fact that he is smoking, during the survey. That is something that D.P. does not promise: unconditional freedom from distinguishing. This is not however a violation of the definition of D.P., as the survey teaches us that specific private attributes correlate with public observable attributes, since this correlation would be observed independent of the presence or absence of the individual in the survey.

There are several more issues as the ones covered above, however we are not going to focus on those, rather on the advantages of D.P.

\chapter{EXAMINATION OF PREVIOUSLY-EXISTING GLOBAL DP LIBRARIES}

The first goal of this Thesis is to examine previously existing programming libraries and APIs that provide the application of Differential Privacy to a dataset. This has been achieved by many companies, such as Google and IBM, but also from research programs like ARX that study the benefits of data privacy. We separate those implementations regarding their output. The possible outputs of a mechanism that adds D.P. to a dataset can be:
\begin{itemize}
    \item An answer to a query, in a private manner.
    \item An anonymized dataset, that meets the criteria of D.P.
\end{itemize}

In the first category, we can distinguish libraries such as Google's and IBM's, that have functions which if applied on a dataset, and given a specific query, can return a single answer.

In the second one, we can find libraries such as the ARX tool, that given a dataset and a group of privacy settings (such as the amount of noise to be inserted), produces an anonymized version of a dataset, that has obviously reduced information in comparison to the original one, but is usable by the final user.

In this chapter, we are going to test those libraries by providing different kinds of datasets, in order to determine the advantages and the disadvantages of each category.

We are going to conduct all of our testings using noise generated by the \textbf{Laplace Mechanism}, thus we must first define its theoretical behavior.



\section{The Laplace Mechanism}

The Laplace Mechanism is used widely in applications of Differential Privacy regarding \textbf{numerical queries}, which are actually functions that match a query to a number, or a vector of numbers, thus answering to it. The mechanism uses noise produced by the Laplace probabilistic distribution, which is equivalent to the query's sensitivity.

\subsection{Query Sensitivity}
The $l_1$ sensitivity of a query $f$, is defined as following:

\begin{align*}
    \Delta f = \max_{\{||x-y||_1 = 1\}} ||f(x) - f(y)||_1
\end{align*} where $x,y \in N^{|X|}$.

This quantity shows the effect by which a single participant's data can change in the worst case during the query $f$, and thus, the uncertainty that we must insert to to the response in order to protect them.

\subsection{The Laplace Distribution}
The Laplace Distribution with a scale $b$, is the distribution with probability density function: 
\begin{align*}
Lap(x|b) = \frac{1}{2b}exp(-\frac{|x|}{b})
\end{align*}

who's variance is $\sigma^2 = 2b^2$, and is actually a symmetric version of the exponential distribution.

\subsection{Use of Laplace in D.P.}

In order to be of use in our definition, the scale of the noise will be calibrated to the sensitivity of the query $f$, divided by epsilon. Thus, the noise used will be drawn from

\begin{align*}
Lap(\frac{\Delta f}{\epsilon})
\end{align*}

Of course, many other probabilistic distributions can be used to ensure differential privacy, but during our testings we prefer to use Laplace.

\section{Query Answering Libraries}

We are going to begin by testing libraries that belong in the first category, and specifically the \textbf{IBM's diffprivlib}, which is written in python, and is publicly available \href{https://github.com/IBM/differential-privacy-library}{here}. The library includes a host of mechanisms, the building blocks of differential privacy, alongside a number of applications to machine learning and other data analytics tasks. We are going to focus our testings in the simple queries, such as the \textbf{mean value}, the \textbf{extreme values} and the \textbf{histograms} of a numerical dataset. The library consists of three modules:

\begin{itemize}
    \item \textbf{Mechanisms}, as known from the theoretical foundations of D.P.
    \item \textbf{Models}, especially machine learning models, that will not concern us during this thesis
    \item \textbf{Tools} that will allow us to apply D.P. in datasets.
\end{itemize}

We are going to use the tools available, in order to apply differential privacy in a dataset of our own, guided of course by the mechanisms provided by diffprivlib. First, we are going to take a look at the dataset that we are going to use going forward.


\subsection{Setup of the mechanism}

The first step in order to test the library, is to setup the mechanism by defining its properties and parameters. 

\subsubsection{Bounds' Selection}

One of the most important aspects for us if we want to apply DP algorithms, is to define the bounds, i.e. the range that a variable can be in. It would be very convenient in our case to just take the tuple of the smallest and the largest value in the column that we are interested in. However, in the real world, the person who asks the queries is not supposed to know this info about the dataset. Thus, since a solution is not provided by the library, we must define our own bounds by guessing the lowest and the highest values in the fields that we want to examine. 

Thus, the user must have somewhat of a previous knowledge regarding the dataset, in order to decide the minimum and maximum value. Those values do not need to be precise, although the more close they are to real ones, the best the protocol will function. At the same time, we must be sure that during our selection we do not leave some of the dataset's values outside of the bounds, as they will be ignored in the final results. 


\subsubsection{Privacy Budget}

In this form of D.P., someone trying to breach the users' privacy, could theoretically ask an infinite number of questions, and thus each time gain more and more information about their private data. This is not covered by the definition of D.P., however it is not acceptable. In the same manner, an untrusted user of the library could ask the same question many times, aiming to determine how much noise is added each time, in order to find out the actual answer to the query, as the way that the noise is drawn is already known.

In order to eliminate this problem, a special parameter call the \textbf{privacy budget} is implemented. The library offers the ability to initialize this budget before asking any queries. During the queries, this budget is each time decreased, according to how much data the answer to the question reveals. For example, the answer to the "mean value of the charges for a surgery", costs less than the answer to the "histogram values of heart transplant surgeries in the West coast".

This parameter is implemented by the library as the budget accountant. This variable tracks the privacy spent, so that our system is not left exposed after lots of "expensive queries". The system will allow someone to ask one question that uses the whole privacy budget, or a series of questions whose total impact is less or equal to the initial budget.

\subsection{Testings' goal}

When applying D.P. mechanisms to our data, we provide the privacy settings of our choice (epsilon variable), and obtain an answer to each of our questions. Thus, in our testings, our goal is to \textbf{determine the accuracy of the answers}, given a specific ε, or some other settings, and comparing them to the true answer, using some metrics. Those metrics are different for each query type. In this section we are going to focus on two types of queries: statistical, and histograms.

\subsection{General techniques}
In each one of our following testings, we are going to run the query \textbf{many times}. As we already know, D.P. relies on probabilistic algorithms that can some times produce extreme results. This may be rare, but we want our testings and conclusions to be accurate. So, we are going to run each query 100 times, and return as a result the mean value of those runs.

\subsection{Statistical Queries}

The first type of queries that we are going to test are those that answer questions like "What is the mean cost of a surgery?", or "What is the largest fee paid by medicare for a transplant?", known as statistical queries. 

\subsubsection{Metrics used}

In the case of statistical queries, their answer is usually a real number, so in order to check their alteration with the true answer, we are going to take into account the \textbf{absolute difference between the truth and the query answer}. 

\subsubsection{Bounds Definition}

We are considering fees for surgeries in our example, thus a logical lower bound would be 0\$ (surgeries could be done pro bono too!), and an upper bound would be 1 million dollars. Either way, we are trying to be extreme with our picks, in order to not find ourselves in the unfortunate situation that a value taken into consideration by the DP query would be out of bounds.

\subsubsection{The identity of the testing Dataset}

The dataset chosen to test the library, is the publicly available "Surgery Charges Across the U.S.", that contains many different kinds of surgeries in a plethora of different hospitals. Our goal is to protect each hospital's data when it comes down to a specific surgery, while helping a patient choose one, depending on the charges that can be found all over the United States. The data provided in this dataset, is going to help a potential patient balance his need of top care, and the need to spend less money. The columns contained in the dataset are:

\begin{itemize}
    \item Surgery code and definition
    \item Provider hospital name
    \item Provider city
    \item Average total payments
    \item Average medicare payments
\end{itemize}

We are going to focus on the last two columns, in order to approximate the charges of a surgery. The above table gives us an image of the containers of the dataset.

The dataset contains a total of 200,000 entries, a more than satisfying number for running D.P. algorithms.

\begin{table}[!htb]

    \caption{"Surgery Charges Across the U.S." dataset columns}
    \label{numbers}

    \begin{tabular}{| c | c | c | c | c| c |}
      \hline 
      ID & Surgery Type & Hospital Name & Hospital City & Total & Medicare \\
      \hline
      1 & TRANSPLANT & MAYO CLINIC & PHOENIX & \$240422.80 & \$133509.55\\
      \hline
      2 & ECMO &  GROSSMONT HOSP & LA MESA & \$193617.86 & \$192003.43 \\
      \hline
      3 & CRANIOTOMY & STANFORD HOSP &  STANFORD & \$32597.87 & \$29347.12  \\
      \hline
    \end{tabular}

\end{table}

\subsubsection{General Dataset Utilities Queries}

Our first experiment is just to ask for some of the utilities of the dataset, and specifically its cost column: the \textbf{mean value}, the \textbf{variance}, the \textbf{sum} and the \textbf{standard deviation} values of the surgeries' cost. 

All of those queries can be executed using the following command (specifically for the mean value query):
\bigskip

\begin{lstlisting}[language=Python]
mean_with_dp = dp.tools.mean(df["Average_Total_Payments"].tolist())
\end{lstlisting}
\bigskip

where the dataframe column is the one containing the cost of each surgery, and its values should be in a list in order for the library to function.

By running the above mentioned queries, we got the following results:


\begin{table}[!htb]
    \centering
    \caption{General Queries results for Surgeries Dataset}
    \label{numbers}

    \begin{tabular}{| c | c | c |}
      \hline 
      Query & True answer & Private Answer \\
      \hline
      Mean value & 13168.5 & 13167.3 \\
      \hline
      Sum & 262754253.1 &  262935459.3 \\
      \hline
      Variance & 262754253.1 & 261940796.5\\
      \hline
      Standard Deviation & 18855.1 & 25825.0\\
      \hline
    \end{tabular}

\end{table}


The answers are almost perfect, for example on the mean value, considering that the cost is thousands of dollars, and the error is just 1 dollar. The simplicity of the query just lies to the following instruction:

However, this is just a simple example, executed only once, hence it can not provide us with safe conclusions for the library. In order to do so, we are going to run more complicated examples moving forward.

\subsubsection{Lowering the Dataset size}

As we mentioned above, the entries that the dataset contains are a very large number, and we know as a fact that D.P. functions well when this is the case. How is the library going to respond though if the dataset size is smaller? We are going to run for 4 different values of epsilon, and get the results for an increasing number of entries, starting from 10, and moving to 2000. Thus, the X axis of each plot represents the increasing dataset size, the Y axis the accuracy error, and each plot has a title of the epsilon setting used for the measurements. We can see the results in the \textbf{Figure 3.1} below.

\begin{figure}[!htb]\centering
    \includegraphics[width=1\textwidth]{images/increasing_ds_size.png}
    \caption{Accuracy Error for Increasing Dataset Sizes}
\end{figure}


By observing the plots, we can make a couple of conclusions:

\begin{itemize}
    \item \textbf{The smaller the epsilon gets, the bigger the accuracy error in the case of small datasets.} This, according to the definition makes sense, because small epsilon indicates higher privacy, thus for small datasets it can mean lower accuracy, due to the high amount of noise added.
    \item \textbf{The accuracy error stabilizes near 0 as the size of the dataset gets over 1000 entries.} Of course, depending to the epsilon value, this point could be earlier in the dataset sizes, as we observe for ε $= 1$. This again lays in the above mentioned property of the definition.
\end{itemize}

\subsubsection{Epsilon measurements}

The most important aspect when applying Differential Privacy to a dataset, is the selection of epsilon. This number is held accountable of the trade-off that DP offers: how much accuracy are we going to sacrifice in order to have less privacy loss, and vice versa. Given the library and our surgeries' costs dataset, we are going to measure the accuracy changes with the selection of different epsilon values.

The size of the dataset is too big, and thus the computation of the sensitivity will take too long. We assume that the data are somewhat equally distributed, thus we chose only 1\% of the dataset (which is a significant number of members), to take part in our sensitivity calculations.

In order to observe if the library functions well, we are going to compare the results produced with the \textbf{theoretical bounds of the Laplace mechanism}. 

In theory, the accuracy of an ε-differential private query, can be at most equal to the sensitivity of the query divided by epsilon, thus $$\frac{\Delta f}{\epsilon}$$ where $\Delta f$ denotes the sensitivity, and ε is our current privacy setting. 

Sensitivity is defined as the maximum difference that can be found if we alter a single entry in the dataset. We can define 2 types of sensitivity:

\begin{itemize}
    \item \textbf{Local Sensitivity} which in theory is $ \Delta f = \max_{||x-y||_1 = 1} ||f(x)-f(y)||_1$
    \item \textbf{Global Sensitivity} which we are going to use in our testings, and is defined as:
    \begin{align*}
        \Delta f = \frac{upper - lower}{length(DB)}
    \end{align*} where upper and lower are the highest and lowest values of the column we are interested in our dataset, and length is the size of the database.
\end{itemize}

In order to compute the local sensitivity, we must check the maximum difference in the query result that occurs if we remove a single person from the database. We are going to define a function to do so, so we can check the theoretical bounds during our tests.

The results of those testings are shown in the \textbf{Figure 3.2}.

\begin{figure}[!htb]\centering
    \includegraphics[width=1\textwidth]{images/epsilon_measurements.png}
    \caption{Real and Optimal Accuracy Error for Increasing epsilon values}
\end{figure}

It is clear that as we increase the epsilon value, the privacy loss gets bigger. On the other hand, if epsilon is too small, as we can see in the above plot, extreme errors in accuracy in our queries will emerge. 

The optimal value of epsilon varies, there is no general rule for the perfect epsilon. It depends on many different aspects, such as:

\begin{itemize}
    \item The noise generated by the probabilistic mechanism used
    \item The implementation of the algorithm
    \item The size of the dataset
    \item The query itself.
\end{itemize}

Moreover, the selection of epsilon depends on the dataset. For example, we might have a dataset that is extremely important to have minimal privacy loss. In that case, we will opt to use a rather small epsilon, thus we do not disclose the sensitive data included. An other dataset could be less sensitive, but the analyst might have a need for extreme accuracy every time, so the epsilon selected should be rather big.

Regarding the comparison with the Laplace bounds, we can see that the library's query performs significantly well throughout the different epsilon values selected. This is due to the fact that IBM uses the same formula to compute the sensitivity, as well as a similar mechanism (Laplace truncated noise), in order to answer to our queries.

\subsection{Histogram Queries}

Histogram graphs are a very handy way to visualize numerical data, compare different values of a specific field, and thus extract conclusions about the dataset. We are going to study the `diffprivlib`'s method of creating an histogram, and its accuracy when changing the epsilon factor.

The IBM DP library offers a differential private way to create histograms. The difference with the simple queries that we tested, is that now, \textbf{geometric truncated} noise is added in order to satisfy DP.

\subsubsection{Metrics Used}
The result of a histogram query on a dataset is a vector containing how many entries belong on a specific range. Thus, the comparison between 2 histograms can be held out by comparing those vectors. 

There are plenty of metrics used to compare vectors, but we are going to focus on the \textbf{Euclidean Metric} and the \textbf{Kantorovich metric} (also known as Wasserstein or EMD metric). 

The Euclidean metric is one of the most simple metrics, as it takes into account the distance between each pair of the two vectors. Thus, the distance of the vectors is defined by:

\begin{align*}
    d = \sqrt{\sum_{i=1}^n (x_i - y_i) ^ 2}
\end{align*}

where $n$ are the total elements of the vectors (must be of equal size), and $x$ and $y$ are the 2 vectors that we are comparing.

The Kantorovich metric is a more complex one, as it examines the cost to move a specific quantity of the one vector to the other, in order for the two to be similar, and thus figuring out their distance. For example, the Kant. distance is larger if we have the vectors $[0 0 0 1]$ and $[1 0 0 0]$ than having the vectors $[0 0 0 1]$ and $[0 0 1 0]$, as in the first case the first element of the histogram is moved 3 places in order to be similar to the second one. In order to determine the cost, a base metric is used, which in our case will be the euclidean.

The way that the metric works in a naive approach, is explained in \textbf{Figure 3.3.}

\begin{figure}[!htb]\centering
    \includegraphics[width=0.7\textwidth]{images/emd.png}
    \caption{Kantorovich Metric Application on 2 histograms}
\end{figure}

This metric is much more suitable for D.P. as we are not just interested in the alteration of the results, but on the amount of alteration. If, for example, we are examining a histogram that contains the age of a hospital's patients, it is not the same for the private histogram to deem a 10 year old patient as 90 year old, than deeming him as 11. 

In order for the Kantorovich metric to be computed, we must solve a Dynamic Programming Problem, and make complex calculations, which are beyond the subject of this thesis. There are many implementations of the metric, and since IBM's library is written in Python, we are going to use the QIF library %TODO:Insert mention.
which is also available as a Python library. 


\subsubsection{Bounds Definition}

In the following testings we are examining salaries, thus we must set the lower and the higher bounds such as to support the most extreme salaries that could possibly be present in the dataset. Therefore, the lower bound is set to 0 dollars, and the higher to 500,000 dollars.

\subsubsection{The identity of the testing Dataset}

During our histogram testings, we are going to use a different dataset, which contains  sensitive data regarding employee's salaries in the state of Baltimore, while stating other facts about the members of the dataset. 

The columns contained are: 
\begin{itemize}
    \item Name
    \item Job Title
    \item Annual earnings
    \item Gross earnings
\end{itemize}

We are going to focus on the last 2 columns, containing the employees' salaries. The dataset has 13000 entries, which are more than enough in order for our histograms to be realistic and accurate. The above table gives us an image of the containers of the dataset.

\begin{table}[!htb]
    \centering

    \caption{"Baltimore State Employees Salaries" dataset columns}
    \label{numbers}

    \begin{tabular}{| c | c | c | c |}
      \hline 
      Name & Job Title & Annual Earnings & Gross \\
      \hline
      Aaron,Kareem D & Utilities Inst Repair I	 & 32470.0 & 25743.94 \\
      \hline
      Abadir,Adam O	 & Police Officer &  60200.0 & 57806.13  \\
      \hline
      Abbeduto,Mack & Council Technician & 53640.0 &  59361.55 \\
      \hline
    \end{tabular}
\end{table}

\subsubsection{Simple queries}

At first, we are going to run simple instances of histogram queries, so we can take a look at the amount of error that we expect moving forward, and also familiarize with the execution of the commands. To create a private histogram using the library, we must provide the bins. Those should be identical to our bounds, that we have earlier defined. While creating the non-private histograms we are going to use the same bounds, in order for our comparisons to make sense. The creation of the bins and the private histogram are achieved using the following Python instructions:

\bigskip
\begin{lstlisting}[basicstyle= \footnotesize,
language=Python]
bins = np.linspace(0, 300000, 20)[1:]
result = dp.tools.histogram(input_list, bins = bins, 
         epsilon = epsilon, range=bounds_range)[0]
\end{lstlisting}
\bigskip

where epsilon is the float number selected as the epsilon parameters, and bounds range is the tuple of our dataset bounds.

The result of the execution of both the private and the non-private histogram queries for the annual earnings column are shown below, in the \textbf{Figure 3.4}.

\begin{figure}[!htb]\centering
    \includegraphics[width=1\textwidth]{images/simple_hists.png}
    \caption{Private and Non-private histograms for the Salaries Dataset}
\end{figure}

The results are more than satisfying in the naked eye. This is probably due to the large dataset size: we are not able to locate small changes. In order to do so, we are going to check our error using the accuracy error function that we have defined.

While applying our metrics, the Euclidean error is $14.81$, and the Kantorovich error is $91.63$. The euclidean distance error determines how many entries were wrongly classified in a bin (in average), when the differentially private query was run. So, less than 20 people out of 13.000 were wrongly classified, whereas their privacy was secured. This is quite a good trade-off!

\subsubsection{Epsilon Measurements}

Once again, we are going to run the histogram queries for different values of epsilon, in order to check their behavior as the parameter increases. The results of the execution for the Euclidean metric are shown in the \textbf{Figure 3.4}, and for the Kantorovich metric in the \textbf{Figure 3.5}. 


\begin{figure}[!htb]\centering
    \includegraphics[width=1\textwidth]{images/hist_metrics_euclidean.png}
    \caption{Euclidean Metric Error for increasing values of epsilon}
\end{figure}

\begin{figure}[!htb]\centering
    \includegraphics[width=1\textwidth]{images/hist_metrics_kantorovich.png}
    \caption{Kantorovich Metric Error for increasing values of epsilon}
\end{figure}

We observe that the error curves follow the same ratio as the previous ones, indicating what we already know, that the error decreases as the epsilon values increase. This is another case where the library produces accurate and definition-aware results. 

\subsubsection{Histogram queries in theory}
Based on [1] the histogram queries are very high sensitivity queries, thus a slight change to the bounds could be critical for their result. 

The authors suggest that we use noise generated by the LaPlace mechanism, but with a slight change. In detail they suggest the following:

\textit{"In the special (but common) case in which the queries are structurally disjoint we can do much better — we do not necessarily have to let the noise scale with the number of queries. An example is the histogram query. In this type of query the universe $N^X$ is partitioned into cells, and the query asks how many database elements lie in each of the cells. Because the cells are disjoint, the addition or removal of a single database element can affect the count in exactly one cell, and the difference to that cell is bounded by 1, so histogram queries have sensitivity 1 and can be answered by adding independent draws from $Lap(\frac{1}{\epsilon})$ to the true count in each cell."}
\clearpage

\subsection{Conclusions}

After a rather satisfying amount of testing in a large, real world dataset, we can safely say, that the IBM DP library has quite impressive results when it comes down to the trade-off between privacy and accuracy during both simple counting queries and histogram ones. However, we are cautious about 2 problems that were observed while using the library:

\begin{itemize}
 \item \textbf{Bounds checking}. The user must define himself the bounds, a fact that causes for speculations on the variance of the values in the dataset. It would be convenient to take the lowest and highest value in the field that we are examining, in fact that is how IBM demonstrates those examples, but that violates the rule that prevents the user from having any info of the dataset before the DP processing.
 
 \item \textbf{Non-DP preprocessing}. If we ask complicated queries (ex Surgeries performed in Stanford), the library does not offer a way to preprocess the data, thus we trust python in doing so, which results in a non-DP way of shrinking the dataset. The result obtained is of course differential private, but what happens if the dataset has only 1 record in it? That, while being a very extreme case, violates the definition of differential privacy.
\end{itemize}

The bounds' problem can be solved by running a minimum and maximum value differentially private query, which might in some cases leave certain datasets out, but is a rather satisfying solution. 

The Non-DP preprossessing is a more tricky one, that needs special techniques in order to cover the edge case mentioned. However, we are going to check such techniques in the following section, which is based on returning the whole dataset after the application of D.P.



% add main chapters (should be given in capital letters)
\chapter{Introduction}
  \begin{greek}
    \section{Μορφοποίηση Κειμένου}
    Το παρόν αρχείο αποτελεί το υπόδειγμα (template) για τη μορφοποίηση της εργασίας.

    Για την ομοιόμορφη εμφάνιση των σχετικών τόμων και του ψηφιακού υλικού που θα παραδίδονται στην Βιβλιοθήκη, το Τμήμα καθιερώνει υποχρεωτικά πρότυπα. Για το σκοπό αυτό θα πρέπει να τηρούνται αυστηρώς οι οδηγίες που παρατίθενται στη συνέχεια.

    \subsection{Μέγεθος σελίδας}
    Το μέγεθος της σελίδας θα πρέπει να είναι \textbf{A4}.

    \subsection{Ημερομηνίες}
    Τα στοιχεία του μηνός και του έτους που θα αναγράφονται στη εργασία είναι αυτά της ημερομηνίας εξέτασης. Τα ίδια ημερομηνιακά στοιχεία θα αναγράφονται και σε οποιοδήποτε συνοδευτικό υλικό κατατίθεται στη Βιβλιοθήκη.

    \subsection{Εξώφυλλο και \grnumm{1} Εσώφυλλο (Σελίδα Τίτλου)}
    Όπως στην αρχή του παρόντος προτύπου. Δηλαδή με τη σειρά:
    \begin{enumerate}
      \item Εικονίδιο της Αθηνάς: άνω στο κέντρο.
      \item Τίτλος του Πανεπιστημίου: Arial έντονα κεφαλαία 14.
      \item Τίτλος Σχολής Arial έντονα κεφαλαία 12.
      \item Τίτλος του Τμήματος: Arial έντονα κεφαλαία 12.
      \item Είδος εργασίας (Πτυχιακή Εργασία) Arial έντονα κεφαλαία 12.
      \item Τίτλος της Εργασίας: Arial έντονα πεζά 16.
      \item Όνομα, αρχικό γράμμα πατρώνυμου και επώνυμο φοιτητή: Arial έντονα πεζά 12.
      \item \underline{Μόνο στην περίπτωση των \textbf{Πτυχιακών} και \textbf{Διπλωματικών} εργασιών}

            Επιβλέπων (ή Επιβλέπουσα) ή Επιβλέποντες (ή Επιβλέπουσες): Όνομα και επώνυμο καθηγητή Arial πεζά έντονα 12, τίτλος καθηγητή Arial πεζά 12. Στην περίπτωση που υπάρχουν συνεπιβλέποντες μη μέλη ΔΕΠ προστίθενται στην αμέσως επόμενη γραμμή κατά τον ίδιο τρόπο με τους επιβλέποντες καθηγητές.
      \item Τόπος ολοκλήρωσης της εργασίας (που είναι πάντα ΑΘΗΝΑ): Arial έντονα κεφαλαία 12.
      \item Μήνας και έτος ολοκλήρωσης της εργασίας: Arial έντονα κεφαλαία 12. Θα είναι ο μήνας και το έτος εξέτασης της εργασίας.
      \item Το διάστιχο στα στοιχεία του εξωφύλλου και 1ου εσώφυλλο θα πρέπει να είναι 1pt.
      \item Η αρίθμηση των σελίδων αρχίζει νοητά από το 1ο εσώφυλλο (σελίδα τίτλου), χωρίς όμως να αναγράφεται ο αριθμός της σελίδας σε αυτό. Η αρίθμηση των σελίδων θα αρχίσει να φαίνεται από την 1η σελίδα του πίνακα περιεχομένων και μετά.
      \item Το πίσω μέρος της σελίδας αυτής παραμένει λευκό.
    \end{enumerate}

    \subsection{\grnumm{2} Εσώφυλλο (Σελίδα έγκρισης)}
    Όπως στην αρχή του παρόντος προτύπου. Δηλαδή με τη σειρά:

    Είδος εργασίας: \textbf{ΠΤΥΧΙΑΚΗ ΕΡΓΑΣΙΑ} Arial έντονα κεφαλαία 12

    Τίτλος: Arial πεζά 12

    Κέντρο:
    \begin{itemize}
      \item Όνομα και επώνυμο φοιτητή: Arial έντονα πεζά 12
      \item Αριθμός Μητρώου (Α. Μ.) του φοιτητή (μόνο για τις πτυχιακές και τις μεταπτυχιακές εργασίες): Arial κεφαλαία 1
    \end{itemize}

    Αριστερά:

    «Επιβλέπων (ή Επιβλέπουσα ή Επιβλέποντες ή Επιβλέπουσες)» (για τις πτυχιακές εργασίες).
    \begin{itemize}
      \item Arial έντονα κεφαλαία 12
      \item Τίτλος Καθηγητή: Arial πεζά 12
      \item Όνομα και Επώνυμο Καθηγητή: Arial έντονα πεζά 12
    \end{itemize}

    Το πίσω μέρος της σελίδας αυτής παραμένει λευκό.

    \subsection{Περίληψη}
    Μετά το \grnumm{2} εσώφυλλο θα ακολουθούν σε δύο χωριστά φύλλα η περίληψη της εργασίας στην ελληνική γλώσσα και η περίληψη της εργασίας στην αγγλική. Η περίληψη περιλαμβάνει το σκοπό-αντικείμενο της εργασίας, τη μεθοδολογία, τα κύρια βήματα που ακολουθήθηκαν και τέλος τα κύρια αποτελέσματα.

    Μετά το τέλος της περίληψης θα δηλώνεται η θεματική περιοχή της εργασίας και 5 λέξεις κλειδιά (ελληνικά και αγγλικά αντίστοιχα για κάθε σελίδα). Η συνολική έκταση της περίληψης και των λέξεων δήλωσης επιστημονικής περιοχής και λέξεων κλειδιών θα είναι μέχρι μία σελίδα (δείτε και σελίδες 3 και 4 στο παρόν υπόδειγμα).

    Το πίσω μέρος των σελίδων αυτών παραμένει λευκό.

    Στην περίπτωση της \underline{πτυχιακής} εργασίας, η περίληψη, η επιστημονική περιοχή και οι λέξεις κλειδιά στην αγγλική γλώσσα είναι προαιρετικά.

    Ακολουθούν σε χωριστές σελίδες, όπως και στο παρόν υπόδειγμα:

    \textbf{Αφιερώσεις} \textit{(προαιρετικά - το πίσω μέρος της σελίδας αυτής παραμένει λευκό)}

    \textbf{Ευχαριστίες} \textit{(προαιρετικά - το πίσω μέρος της σελίδας αυτής παραμένει  λευκό)}

    \textbf{Περιεχόμενα}

    \textbf{Πρόλογος} (Όπου αναφέρονται θέματα που δεν είναι επιστημονικά ή τεχνικά, όπως το πλαίσιο που διενεργήθηκε η εργασία, ευχαριστίες, ο τόπος διεξαγωγής κλπ.)

    \subsection{Αρίθμηση σελίδων}
    Η αρίθμηση των σελίδων πάντοτε αρχίζει νοητά από το \grnumm{1} εσώφυλλο (σελίδα τίτλου, η \grnumf{1} σελίδα του παρόντος υποδείγματος) χωρίς δηλαδή να αναγράφεται ο αριθμός της σελίδας σε αυτό. Και στο \grnumm{2} εσώφυλλο (σελίδα έγκρισης) επίσης ο αριθμός της σελίδας υπολογίζεται χωρίς  να αναγράφεται σε αυτό. Η αρίθμηση πάντοτε τελειώνει στην τελευταία τυπωμένη σελίδα. Θα πρέπει να αρχίσει να αναγράφεται ο αριθμός σελίδων από την πρώτη σελίδα του πρώτου κεφαλαίου (όπως στο παρόν υπόδειγμα).

    \subsection{Οι σελίδες του Κειμένου}
    Όπως στο παρόν υπόδειγμα. Δηλαδή:
    \begin{itemize}
      \item \textbf{Περιθώρια (Margins):}
      \begin{enumerate}[$\circ$]
        \item Άνω (Top): 2 cm
        \item Κάτω (Bottom): 2 cm
        \item Περιθώριο Βιβλιοδεσίας (Gutter): 0.5 cm
        \item Αριστερά (Left): 2 cm
        \item Δεξιά (Right): 2 cm
      \end{enumerate}
      \item \textbf{Κεφαλίδα (Header):} 1.25 cm (από πάνω): Ο τίτλος της εργασίας \emph{(δεν εισάγεται κεφαλίδα στο εξώφυλλο, στο 1ο και 2ο εσώφυλλο, στις σελίδες των περιλήψεων, στις σελίδες των αφιερώσεων και των ευχαριστιών και στις τυχόν λευκές σελίδες)}.
      \item \textbf{Υποσέλιδο (Footer):} 1.25 cm (από κάτω): Το όνομα ή τα ονόματα των συγγραφέων και ο αριθμός σελίδας \emph{(δεν εισάγεται υποσέλιδο στο εξώφυλλο, στο 1ο και 2ο εσώφυλλο, στις σελίδες των περιλήψεων, στις σελίδες των αφιερώσεων και των ευχαριστιών και στις τυχόν λευκές σελίδες)}.
      \item \textbf{Αρίθμηση σελίδας:} Δεξιά του υποσέλιδου και στην περίπτωση εκτύπωσης και από τις δύο πλευρές του φύλλου στο κέντρο του υποσέλιδου (Προσοχή: στο παρόν υπόδειγμα η αρίθμηση έχει γίνει στα δεξιά του υποσέλιδου, για εκτύπωση στη μία σελίδα του φύλλου). Το παρόν πρόπυπο υποστηρίζει την επιλογή \texttt{dualpage}, η οποία ρυθμίζει το υποσέλιδο για εκτύπωση και στις δύο πλευρές του φύλλου. Το μέγεθος γραμματοσειράς για την αρίθμηση της σελίδας θα πρέπει να είναι 10.
      \item \textbf{Μορφή Παραγράφου (Format Paragraph)}
      \begin{enumerate}[$\circ$]
        \item \textbf{Στοίχιση (Justification):} αριστερά και δεξιά
        \item \textbf{Διάκενο μεταξύ παραγράφων (paragraph spacing):}

        πριν: 0 στιγμές, μετά: 6 στιγμές
        \item \textbf{Διάστιχο (Line spacing):} 1 γραμμή
      \end{enumerate}
      \item \textbf{Γραμματοσειρά (Font):} Arial 12.
      \item \textbf{Τύπος Γραμματοσειράς (Font style):} Normal ή Regular.
      \item \textbf{Αρίθμηση Κεφαλαίων:} Arial 12 ή 14.
      \item \textbf{Τύπος Αρίθμησης Κεφαλαίων:} όπως στο παρόν υπόδειγμα.
      \item \textbf{Τίτλος Κεφαλαίων:} Κεφαλαία έντονα Arial 14, στοίχιση στο κέντρο.
      \item \textbf{Τίτλος Υποκεφαλαίων:} Έντονα (Bold) πεζά Arial 12, στοίχιση αριστερά.
      \item \textbf{Σχήματα/Διαγράμματα:} Κάθε σχήμα/διαγράμμα θα πρέπει να έχει υποχρεωτικά μοναδική αρίθμηση, είτε στο σύνολο της εργασίας είτε ανά κεφάλαιο, και οπωσδήποτε λεζάντα στο κάτω μέρος τους (τα διαγράμματα ανήκουν στην κατηγορία των σχημάτων), με στοίχιση όπως στο παρόν υπόδειγμα.
      \item \textbf{Εικόνες/Φωτογραφίες:} Όλες οι εικόνες/φωτογραφίες θα πρέπει να έχουν υποχρεωτικά μοναδική αρίθμηση και οπωσδήποτε λεζάντα στο κάτω μέρος τους, όπως στο παρόν υπόδειγμα.
      \item \textbf{Πίνακες:} Όλοι οι πίνακες πρέπει να φέρουν μοναδική αρίθμηση και λεζάντα στο πάνω μέρος τους, όπως στο παρόν υπόδειγμα.
    \end{itemize}

    \subsection{Ορολογία}
    Σε περίπτωση συγγραφής στα Ελληνικά, την πρώτη φορά που θα εμφανίζεται στο κείμενο ένας επιστημονικός όρος ο οποίος προέρχεται από μεταφρασμένο ξένο όρο θα αναφέρεται δίπλα σε παρένθεση ο αντίστοιχος ξενόγλωσσος όρος. Στο τέλος του κειμένου θα υπάρχει πίνακας ορολογίας με τις αντιστοιχίσεις των ελληνικών και ξενόγλωσσων όρων. Ως παράδειγμα παράθεσης ορολογίας δίνεται η εξής πρόταση: Ήδη από το 1994 η BELL ξεκίνησε στα εργαστήρια της προσπάθειες για τη σχεδίαση υπολογιστών µε αυξημένη αξιοπιστία (reliability). Δείτε και τον Πίνακα Ορολογίας στο παρόν υπόδειγμα.

    \subsection{Συντμήσεις - Αρκτικόλεξα}
    Στο τέλος του κειμένου θα υπάρχει «Πίνακας Συντμήσεων – Αρκτικόλεξων» όπου θα αναφέρονται οι συντμήσεις-αρκτικόλεξα και δίπλα ή πλήρη ανάπτυξη των ονομασιών. Αν, για παράδειγμα χρησιμοποιήσετε τον όρο W3C στο κείμενό σας, θα πρέπει να παραθέσετε την πλήρη ανάπτυξή του όπως στον Πίνακα Συντμήσεων – Αρκτικόλεξων στο παρόν υπόδειγμα.

    \subsection{Βιβλιογραφικές Αναφορές}
    Οι βιβλιογραφικές αναφορές έχουν ρυθμιστεί στο παρόν πρότυπο, έτσι ώστε να συμφωνούν με τις υποδείξεις του IEEE.

    Μέσα στο κείμενο, οι αναφορές γίνονται γράφοντας \verb!\cite{X}!, όπου \texttt{X} είναι το αναγνωριστικό της πηγής.

    \section{Άλλες Παρατηρήσεις}
    Θα πρέπει να ακολουθείτε το παρόν υπόδειγμα, όσον αφορά τη μορφοποίηση (εξώφυλλα, εσώφυλλα, κλπ) της εργασίας, τις κενές σελίδες, τα περιθώρια της σελίδας, της κεφαλίδας και του υποσέλιδου, τη μορφή της παραγράφου και των γραμματοσειρών, τις λεζάντες σε σχήματα, εικόνες και πίνακες, τη μοναδική αρίθμηση των λεζάντων και ό,τι άλλο εμφανίζεται στο παρόν υπόδειγμα. Επιπλέον, ιδιαίτερη προσοχή δώστε και στις παρακάτω παρατηρήσεις.

    \subsection{Λεζάντες}
    Κάθε σχήμα, διάγραμμα, εικόνα, φωτογραφία και πίνακας θα πρέπει να έχει υποχρεωτικά μοναδική αρίθμηση, είτε στο σύνολο της εργασίας είτε ανά κεφάλαιο, και οπωσδήποτε λεζάντα, όπως φαίνεται πιο πάνω, στο παρόν υπόδειγμα. \textbf{Προσοχή:} για τους πίνακες, η λεζάντα θα πρέπει να βρίσκεται επάνω από τον πίνακα.

    \subsection{Κεφαλίδες και Υποσέλιδα}
    \textbf{Δεν εισάγονται} στο εξώφυλλο, στο \grnumm{1} και \grnumm{2} εσώφυλλο, στις σελίδες των περιλήψεων, στις σελίδες των αφιερώσεων και των ευχαριστιών και στις τυχόν λευκές σελίδες. Εισάγονται από την \grnumf{1} σελίδα του 1ου κεφαλαίου και μετά

  \end{greek}


\chapter{BACKGROUND AND RELATED WORK}
  \lipsum[4-7]
  \begin{figure}
    \centering
    \includegraphics{athena-black.jpeg}
    \caption{The emblem picturing Athena}
    \label{athena}
  \end{figure}

\chapter{ANOTHER CHAPTER}
  \lipsum[10-20]

  \begin{table}
    \centering

    \caption{Some numbers in a table}
    \label{numbers}

    \begin{tabular}{| c | c |}
      \hline
      1 & one \\
      \hline
      2 & two \\
      \hline
      3 & three \\
      \hline
    \end{tabular}

  \end{table}

\chapter{CONCLUSIONS AND FUTURE WORK}
  \lipsum[3-5]
  One reference to some paper \cite{manning:corenlp} and to another paper
  \cite{pennington:glove}.

\backmatter

% abbreviations table
\abbreviations
\begin{center}
	\renewcommand{\arraystretch}{1.5}
	\begin{longtable}{| l | @{\qquad} l |}
	\hline
	RDF    & Resource Description Framework \\
  \hline
	SPARQL & SPARQL Protocol and RDF Query Language \\
  \hline
	OWL    & Web Ontology Language \\
  \hline
	OGC    & Open Geospatial Consortium \\
	\hline
	\end{longtable}
\end{center}

% appendix
\begin{appendix}
% mark the beginning of the appendix
\appendixstartedtrue

% add appendix line to ToC
\phantomsection
\addcontentsline{toc}{chapter}{APPENDICES}

\chapter{FIRST APPENDIX}
\chapter{SECOND APPENDIX}
\chapter{THIRD APPENDIX}
\end{appendix}

% manually include the bibliography
\bibliographystyle{plain}
{\footnotesize \bibliography{references}}

% include it also in ToC (do sth on your own)
\addcontentsline{toc}{chapter}{REFERENCES}

\end{document}
